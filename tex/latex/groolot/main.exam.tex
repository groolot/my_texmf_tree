% Devoir Titre
%
% Auteur  : Gregory DAVID
%%

\documentclass[12pt,a4paper,oneside,titlepage,final]{exam}

\newcommand{\MONTITRE}{Titre}
\newcommand{\MONSOUSTITRE}{Devoir Surveill\'e}
\newcommand{\DISCIPLINE}{\glsentrytext{SiUN} -- \glsentrydesc{SiUN}}
\usepackage[%
	pdftex,%
	pdfpagelabels=true,%
	pdftitle={\MONTITRE},%
	pdfauthor={Gr\'egory DAVID},%
	pdfsubject={\MONSOUSTITRE},%
        colorlinks,%
]{hyperref}
\usepackage{groolot_layout}
\usepackage{groolot_commands}
\usepackage{groolot_glossaire}

\title{
	\begin{tabular*}{\linewidth}{@{\extracolsep{\fill}}lr}
		& {\Huge {\bf \MONTITRE}} \\
		& {\huge \MONSOUSTITRE} \\
		& {\large \DISCIPLINE} \\
	\end{tabular*}
}
\author{
	\begin{tabular*}{\linewidth}{@{\extracolsep{\fill}}lr}
		\multicolumn{2}{r}{version en date du \gitAuthorIsoDate, \REVISIONS}\\
		\hline
		sujet compos\'e {\ifprintanswers\ et corrig\'e\fi} par& \\
                \href{http://www.groolot.net}{Gr\'egory \bsc{David}} & \\
		& \\
		& \\
		& \\
		& \\
		& \\
		& \\
		& \\
		& \\
		& \\
		& \\
		& \\
		& \\
		& \\
		Date : \hrulefill & \\
		Nom : \hrulefill & \\
		Pr\'enom : \hrulefill &
	\end{tabular*}
}

\printanswers\groolotPhiligranne{CORRECTION}

\renewcommand{\HEADER}[2]{
	\pagestyle{headandfoot}
	\thispagestyle{headandfoot}
	\rhead[Nom : \hbox to 4cm{\hrulefill}\\Pr\'enom : \hbox to 4cm{\hrulefill}]{Nom : \hbox to 4cm{\hrulefill}}
	\lhead{\bsc{#1}}
	\headrule
	\lfoot{#2}\cfoot{}\rfoot{\thepage}
	\footrule
}

\begin{document}
\maketitle

\HEADER{\MONTITRE}{\DISCIPLINE} \extrawidth{0.5cm}
\pointsinrightmargin \marginpointname{~\points} \bracketedpoints
\addpoints \qformat{Question
  \thequestion\dotfill\emph{(\pointsofquestion{\arabic{question}}
    \points)}} \bonusqformat{Question
  \thequestion\dotfill\emph{(\bonuspointsofquestion{\arabic{question}}
    \points)}} \setlength\answerlinelength{10cm}

\vspace*{\fill}

% Boite d'information
\begin{center}
    \fbox{\fbox{\parbox{5.5in}{\centering Ce dossier est constitu\'e
          d'un ensemble de \textbf{\numquestions\ questions} comptant
          sur un total de \textbf{\numpoints\ \points}. Vous disposez
          de \textbf{$ \overline{10}_2 $ heures}.\newline ~\newline
          \textbf{Toutes les r\'eponses devront \^etre r\'edig\'ees
            sur le sujet, dans les espaces pr\'evus \`a cet effet, ou
            au dos de la feuille le cas \'ech\'eant.}\newline ~\newline
          \textbf{L'utilisation d'un quelconque mat\'eriel
            \'electronique (calculatrice, t\'el\'ephone mobile,
            ordinateur, etc.) est INTERDIT pour ce devoir.}  }}}
\end{center}

% Affichage du tableau d'\'evaluation
\begin{center}
    \chqword{Questions :} \chpword{Points :} \chbpword{Points bonus :}
    \chsword{\textbf{\'Evaluation :}} \chtword{\textbf{Somme}}
    % \gradetable[h][questions]
    \combinedgradetable[h]
\end{center}

\vspace*{\fill}

\newpage

\begin{questions}
    \question[1] Pr\'esentation, respect des formalismes.

    \question Donner et \gls{demontrer} blabla

    \bonusquestion[1] (cocher une seule solution) :
    \begin{checkboxes}
        \CorrectChoice exacte \choice approxim\'ee \choice vous me
        fatiguez avec vos questions
    \end{checkboxes}

\end{questions}

\printbibheading
% \printbibliography[nottype=online,check=notonline,heading=subbibliography,title={Bibliographiques}]
\printbibliography[check=online,heading=subbibliography,title={Webographiques}]

\printglossaries

\end{document}

