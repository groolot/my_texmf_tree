% Devoir Numération & Codage
%
% Auteur  : Gregory DAVID
%%

\documentclass[12pt,a4paper,oneside,titlepage,final]{exam}

\newcommand{\MONTITRE}{Num\'eration et codage}
\newcommand{\MONSOUSTITRE}{Devoir Surveill\'e}
\newcommand{\DISCIPLINE}{\glsentrytext{SiUN} -- \glsentrydesc{SiUN}}
\usepackage[%
	pdftex,%
	pdfpagelabels=true,%
	pdftitle={\MONTITRE},%
	pdfauthor={Gr\'egory DAVID},%
	pdfsubject={\MONSOUSTITRE},%
        colorlinks,%
]{hyperref}
\usepackage{groolot_glossaire}
\usepackage{groolot_layout}
\usepackage{groolot_commands}

\title{
	\begin{tabular*}{\linewidth}{@{\extracolsep{\fill}}lr}
		& {\Huge {\bf \MONTITRE}} \\
		& {\huge \MONSOUSTITRE} \\
		& {\large \DISCIPLINE} \\
	\end{tabular*}
}
\author{
	\begin{tabular*}{\linewidth}{@{\extracolsep{\fill}}lr}
		\multicolumn{2}{r}{version en date du \gitAuthorIsoDate, \REVISIONS}\\
		\hline
		sujet composé {\ifprintanswers\ et corrigé\fi} par& \\
                \href{http://www.groolot.net}{Gr\'egory \bsc{David}} & \\
		& \\
		& \\
		& \\
		& \\
		& \\
		& \\
		& \\
		& \\
		& \\
		& \\
		& \\
		& \\
		& \\
		Date : \hrulefill & \\
		Nom : \hrulefill & \\
		Pr\'enom : \hrulefill &
	\end{tabular*}
}

\printanswers\groolotPhiligranne{CORRECTION}
\renewcommand{\HEADER}[2]{
	\pagestyle{headandfoot}
	\thispagestyle{headandfoot}
	\rhead[Nom : \hbox to 4cm{\hrulefill}\\Pr\'enom : \hbox to 4cm{\hrulefill}]{Nom : \hbox to 4cm{\hrulefill}}
	\lhead{\bsc{#1}}
	\headrule
	\lfoot{#2}\cfoot{}\rfoot{\thepage}
	\footrule
}

\begin{document}
	% Page de titre
	\maketitle

	% Contenu
	\HEADER{\MONTITRE}{\DISCIPLINE}
	    \extrawidth{0.5cm}
		\pointsinrightmargin
		%\pointsdroppedatright
		\marginpointname{~\points}
		\bracketedpoints
		%\totalformat{Question~\thequestion~:~[\totalpoints~\points]}
		\addpoints
		\qformat{Question \thequestion\dotfill\emph{(\pointsofquestion{\arabic{question}} \points)}}
                \bonusqformat{Question \thequestion\dotfill\emph{(\bonuspointsofquestion{\arabic{question}} \points)}}
		\setlength\answerlinelength{10cm}

\vspace*{\fill}

    % Affichage de la boite d'information
\begin{center}
    \fbox{\fbox{\parbox{5.5in}{\centering Ce dossier est constitu\'e
          d'un ensemble de \textbf{\numquestions\ questions} comptant
          sur un total de \textbf{\numpoints\ \points}. Vous disposez
          de \textbf{$ \overline{10}_2 $ heures}.\newline ~\newline
          \textbf{Toutes les r\'eponses devront \^etre r\'edig\'ees
            sur le sujet, dans les espaces pr\'evus \`a cet effet, ou
            au dos de la feuille le cas échéant.}\newline ~\newline
          \textbf{L'utilisation d'un quelconque mat\'eriel
            \'electronique (calculatrice, t\'el\'ephone mobile,
            ordinateur, etc.) est INTERDIT pour ce devoir.}  }}}
\end{center}

% Affichage du tableau d'\'evaluation
\begin{center}
    \chqword{Questions :}
    \chpword{Points :}
    \chbpword{Points bonus :}
    \chsword{\textbf{Évaluation :}}
    \chtword{\textbf{Somme}}
    % \gradetable[h][questions]
    \combinedgradetable[h]
\end{center}

\vspace*{\fill}

\newpage

\begin{questions}
    \question[1] Présentation, respect des formalismes.

    \question Donner et \gls{demontrer} démontrer la conversion de base des valeurs
    suivantes :
    \begin{parts}
        \part[1]
        $ \overline{2681}_{10} = \overline{???}_{2} $
        \begin{solution}[6cm]
            $ \overline{2681}_{10} = \overline{101001111001}_{2} $
            \DivisionEuclidienneSuccessive{2681}{2}
        \end{solution}

        \part[1]
        $ \overline{53}_{10} = \overline{???}_{2} $
        \begin{solution}[6cm]
            $ \overline{53}_{10} = \overline{110101}_{2} $
            \DivisionEuclidienneSuccessive{53}{2}
        \end{solution}
        
        \part[1]
        $ \overline{10000101001}_{2} = \overline{???}_{10} $
        \begin{solution}[6cm]
            $ \overline{10000101001}_{2} = \overline{1 \times 2^{10} +
              1 \times 2^{5} + 1 \times 2^{3} + 1 \times 2^{0}}_{10} $
            \newline $ \overline{10000101001}_{2} =
            \overline{1065}_{10} $
        \end{solution}
        
        \part[1]
        $ \overline{128}_{10} = \overline{???}_{16} $
        \begin{solution}[6cm]
            $ \overline{128}_{10} = \overline{80}_{16} $
            \DivisionEuclidienneSuccessive{128}{16}
        \end{solution}

        \part[1]
        $ \overline{136912}_{10} = \overline{???}_{16} $
        \begin{solution}[6cm]
            $ \overline{136912}_{10} = \overline{216D0}_{16} $
            \DivisionEuclidienneSuccessive{136912}{16}
        \end{solution}

        \part[1]
        $ \overline{101111}_{2} = \overline{???}_{16} $
        \begin{solution}[6cm]
            $ \overline{101111}_{2} = \overline{2F}_{16} $
        \end{solution}

        \part[1]
        $ \overline{11111110}_{2} = \overline{???}_{16} $
        \begin{solution}[6cm]
            $ \overline{11111110}_{2} = \overline{FE}_{16} $
        \end{solution}
    \end{parts}

    \question Réaliser, et démontrer, les calculs sur les nombres
    binaires suivants :
    \begin{parts}
        \part L'addition
        \begin{subparts}
            \subpart[1] $ \overline{100111}_{2} + \overline{101}_{2} =
            \overline{???}_{2}$
            \begin{solution}[6cm]
                \centering
                \begin{tabular}{lr}
                    & \texttt{100111} \\
                    $ + $ & \texttt{101} \\ \hline
                    & \texttt{101100} \\
                \end{tabular}
            \end{solution}

            \subpart[1] $ \overline{111}_{2} + \overline{1}_{2} =
            \overline{???}_{2}$
            \begin{solution}[6cm]
                \centering
                \begin{tabular}{lr}
                    & \texttt{111} \\
                    $ + $ & \texttt{1} \\ \hline
                    & \texttt{1000} \\
                \end{tabular}
            \end{solution}

            \subpart[1] $ \overline{1010101}_{2} + \overline{111}_{2}
            = \overline{???}_{2}$
            \begin{solution}[6cm]
                \centering
                \begin{tabular}{lr}
                    & \texttt{1010101} \\
                    $ + $ & \texttt{111} \\ \hline
                    & \texttt{1011100} \\
                \end{tabular}
            \end{solution}
        \end{subparts}

        \part La soustraction
        \begin{subparts}
            \subpart[1] $ \overline{10}_{2} - \overline{1}_{2} =
            \overline{???}_{2}$
            \begin{solution}[6cm]
                \centering
                \begin{tabular}{lr}
                    & \texttt{10} \\
                    $ - $ & \texttt{1} \\ \hline
                    & \texttt{1} \\
                \end{tabular}
            \end{solution}

            \subpart[1] $ \overline{111}_{2} - \overline{10}_{2} =
            \overline{???}_{2}$
            \begin{solution}[6cm]
                \centering
                \begin{tabular}{lr}
                    & \texttt{111} \\
                    $ - $ & \texttt{10} \\ \hline
                    & \texttt{101} \\
                \end{tabular}
            \end{solution}

            \subpart[1] $ \overline{1001001101}_{2} -
            \overline{10111}_{2} = \overline{???}_{2}$
            \begin{solution}[6cm]
                \centering
                \begin{tabular}{lr}
                    & \texttt{1001001101} \\
                    $ - $ & \texttt{10111} \\ \hline
                    & \texttt{1000110110} \\
                \end{tabular}
            \end{solution}

            \subpart[1] $ \overline{110011}_{2} - \overline{1100}_{2}
            = \overline{???}_{2}$
            \begin{solution}[6cm]
                \centering
                \begin{tabular}{lr}
                    & \texttt{110011} \\
                    $ - $ & \texttt{1100} \\ \hline
                    & \texttt{100111} \\
                \end{tabular}
            \end{solution}

            \subpart[1] $ \overline{10000}_{2} - \overline{1111}_{2} =
            \overline{???}_{2}$
            \begin{solution}[6cm]
                \centering
                \begin{tabular}{lr}
                    & \texttt{10000} \\
                    $ - $ & \texttt{1111} \\ \hline
                    & \texttt{1} \\
                \end{tabular}
            \end{solution}
        \end{subparts}

        \part La multiplication
        \begin{subparts}
            \subpart[1] $ \overline{101011}_{2} \times
            \overline{10}_{2} = \overline{???}_{2}$
            \begin{solution}[6cm]
                \centering
                \begin{tabular}{lr}
                    & \texttt{101011} \\
                    $\times$ & \texttt{10} \\ \hline
                    & \texttt{101011.} \\ \hline
                    & \texttt{1010110} \\
                \end{tabular}
            \end{solution}

            \subpart[1] $ \overline{101011}_{2} \times
            \overline{1000}_{2} = \overline{???}_{2}$
            \begin{solution}[6cm]
                \centering
                \begin{tabular}{lr}
                    & \texttt{101011} \\
                    $\times$ & \texttt{10000} \\ \hline
                    & \texttt{101011....} \\ \hline
                    & \texttt{1010110000} \\
                \end{tabular}
            \end{solution}

            \subpart[1] $ \overline{101011}_{2} \times
            \overline{10111}_{2} = \overline{???}_{2}$
            \begin{solution}[6cm]
                \centering
                \begin{tabular}{lr}
                    & \texttt{101011} \\
                    $\times$ & \texttt{10111} \\ \hline
                    & \texttt{101011} \\
                    & \texttt{101011.} \\
                    & \texttt{101011..} \\
                    & \texttt{101011....} \\ \hline
                    & \texttt{1111011101} \\
                \end{tabular}
            \end{solution}

            \subpart[1] $ \overline{11111}_{2} \times
            \overline{11111}_{2} = \overline{???}_{2}$
            \begin{solution}[6cm]
                \centering
                \begin{tabular}{lr}
                    & \texttt{11111} \\
                    $\times$ & \texttt{11111} \\ \hline
                    & \texttt{11111} \\
                    & \texttt{11111.} \\
                    & \texttt{11111..} \\
                    & \texttt{11111...} \\
                    & \texttt{11111....} \\ \hline
                    & \texttt{1111000001} \\
                \end{tabular}
            \end{solution}
        \end{subparts}
    \end{parts}

    \question[20] Donner, démontrer et vérifier la repr\'esentation
    binaire du nombre d\'ecimal suivant sur \lstinline{32bits}, selon
    la norme \texttt{\acrshort{IEEE} 754}\cite{IEEE754_WIKIPEDIA} :
        $$ \overline{-8.75000095367431640625}_{10} = \overline{????????????????????????????????}_{2} $$
        \begin{solution}[30cm]
                $$ \overline{-8.75000095367431640625}_{10} = \overline{11000001000011000000000000000001}_{2} $$
                \begin{proof}~\newline
                    \begin{enumerate}
                        \item conversion binaire de la partie
                        enti\`ere :
                    $$ \overline{8}_{10} = 1 \times 2^{3} = \overline{1000}_{2}$$
                    \item conversion binaire de la partie d\'ecimale :
                    \subitem la mantisse peut contenir
                    \lstinline{23bits}, or la partie enti\`ere
                    normalis\'ee occupe d\'ej\`a \lstinline{3bits} sur
                    l'espace de la mantisse, il nous reste donc
                    \lstinline{20bits} disponibles pour la partie
                    d\'ecimale : $$ 0.75000095367431640625 \times
                    2^{20} = 786433 $$ avec l'arrondi \`a l'entier le
                    plus proche, \subitem nous n'avons plus qu'\`a
                    convertir la valeur $ \overline{786433}_{10} $ en
                    binaire \DivisionEuclidienneSuccessive{786433}{2}
                    \item r\'ecriture compl\`ete du nombre d\'ecimale
                    en binaire
                        $$ \overline{-8.75000095367431640625}_{10} = \overline{-1000.11000000000000000001}_{2}$$
                        \item le nombre n'est pas normalis\'e :
                        normalisation\newline on veut le nombre sous
                        la forme $ S \times 1.M \times 2^{e} $
                        \begin{multline}
                            \overline{-1000.11000000000000000001}_{2}
                            \\= \overline{-1 \times
                              1.00011000000000000000001}_{2} \times
                            2^{3}
                        \end{multline}
                        o\`u
                $$ S = \overline{1}_{2} \textnormal{, car le signe est n\'egatif} $$
                $$ e = 3 \Rightarrow E = e + 127 = 130 = \overline{10000010}_{2} $$
                $$ M = \overline{00011000000000000000001}_{2} $$
                \item vérification :
                \begin{multline}
                    \overline{-1000.11000000000000000001}_{2} \\= -
                    (2^3 + 2^{-1} + 2^{-2} + 2^{-20}) \\= - (8 + 0.5
                    +0.25+0.00000095367431640625) \\=
                    \overline{-8.75000095367431640625}_{10}
                \end{multline}
            \end{enumerate}
        \end{proof}
    \end{solution}

    \bonusquestion[1] La représentation binaire, selon la norme
    \texttt{\acrshort{IEEE} 754} sur \lstinline{32bits}, du nombre précédent ($ -8.75000095367431640625 $)
    est-elle (cocher une seule solution) :
    \begin{checkboxes}
        \CorrectChoice exacte
        \choice approximée
        \choice impossible sur \lstinline{32bits}
        \choice impossible sur \lstinline{64bits}
        \choice improbable sur \lstinline{32bits}
        \choice improbable sur \lstinline{64bits}
        \choice vous me fatiguez avec vos questions
    \end{checkboxes}

    \bonusquestion[2] Expliquer pourquoi.
    \begin{solutionorlines}[10cm]
        La représentation de la valeur décimale en binaire est exacte
        car c'est une somme exacte de puissances de 2.
        $$ -(2^3 + 2^{-1} + 2^{-2} + 2^{-20}) = - (8 +0.5+0.25+0.00000095367431640625) $$
        En effet, lors du codage en binaire de la partie décimale du
        nombre $$ -8.75000095367431640625 $$ il n'a pas été nécessaire
        de faire un arrondi quelconque, sur les \lstinline{20bits}
        restant de la mantisse à notre disposition.

        Ainsi, ce nombre là \textbf{a une représentation exacte} en
        binaire selon la norme \texttt{\acrshort{IEEE} 754} sur
        \lstinline{32bits}.
    \end{solutionorlines}
\end{questions}

\printbibheading
% \printbibliography[nottype=online,check=notonline,heading=subbibliography,title={Bibliographiques}]
\printbibliography[check=online,heading=subbibliography,title={Webographiques}]

\printglossaries

\end{document}

