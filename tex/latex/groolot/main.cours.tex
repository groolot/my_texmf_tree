% Titre
%
% Auteur : Grégory DAVID
%%

\documentclass[12pt,a4paper,oneside,titlepage,final]{article}

\usepackage{style/layout}
\usepackage{style/glossaire}

\newcommand{\MONTITRE}{Titre ou sujet du document}
\newcommand{\MONSOUSTITRE}{sous titre}
\newcommand{\DISCIPLINE}{\glsentrytext{SiUN} -- \glsentrydesc{SiUN}}

\usepackage[%
pdftex,%
pdfpagelabels=true,%
pdftitle={\MONTITRE},%
pdfauthor={Grégory DAVID},%
pdfsubject={\MONSOUSTITRE},%
colorlinks,%
%hidelinks,%
]{hyperref} \usepackage{style/commands}

\title{%
  \begin{flushright}
    \noindent{\Huge {\bf \MONTITRE}} \\
    \noindent{\huge \MONSOUSTITRE} \\
    \noindent{\large \DISCIPLINE} \\
  \end{flushright}%
}

\begin{document}
% Page de titre
\maketitle

% Copyright
\HEADER{Copyright}{Copyleft}
\vspace*{\fill}
\begin{quote}\en
    Copyright \copyright\ \the\year\ \MAILGREGORY.  Permission is
    granted to copy, distribute and/or modify this document under the
    terms of the GNU Free Documentation License, Version 1.3 or any
    later version published by the Free Software Foundation; with no
    Invariant Sections, no Front-Cover Texts and no Back-Cover Texts.
    A copy of the license is included in the section entitled "GNU
    Free Documentation License", or available on the Internet:
    \url{https://www.gnu.org/licenses/fdl.html}.
\end{quote}
\vspace{2cm}
\begin{quote}\fr
    Copyright \copyright\ \the\year\ \MAILGREGORY.  Permission vous
    est donn\'ee de copier, distribuer et/ou modifier ce document
    selon les termes de la Licence GNU Free Documentation License,
    Version 1.3 ou ult\'erieure publi\'ee par la Free Software
    Foundation ; sans section inalt\'erable, sans texte de premi\`ere
    page de couverture et sans texte de derni\`ere page de
    couverture. Une copie de cette Licence est incluse dans la section
    appel\'ee \og GNU Free Documentation License \fg~ de ce document,
    ou disponible sur l'Internet à l'adresse :
    \url{https://www.gnu.org/licenses/fdl.html}.
\end{quote}
\vspace*{\fill}


% Contenu
\HEADER{\MONTITRE}{\DISCIPLINE}

\section{Contexte}\label{sec:contexte}

\section{Travail à faire}\label{sec:travail.a.faire}

\section{Ressources}\label{sec:ressources}

\clearpage
\section{Couverture du référentiel de certification
  \gls{SIO}}\label{sec:couverture.referentiel.sio}

Le contenu de ce document permet d'aborder ou de couvrir les activités
et compétences suivantes :
\begin{enumerate}
  \item [\textbf{A1.1.1}] Analyse du cahier des charges d'un service à
  produire
  \begin{itemize}
    \item [\textbf{C1.1.1.1}] Recenser et caractériser les contextes
    d'utilisation, les processus et les acteurs sur lesquels le
    service à produire aura un impact
    \item [\textbf{C1.1.1.2}] Identifier les fonctionnalités attendues
    du service à produire
    \item [\textbf{C1.1.1.3}] Préparer sa participation à une réunion
  \end{itemize}
  \item [\textbf{A1.1.2}] Étude de l'impact de l'intégration d'un
  service sur le système informatique
  \begin{itemize}
    \item \item [\textbf{C1.1.2.1}] Analyser les interactions entre
    services
    \item [\textbf{C1.1.2.2}] Recenser les composants de
    l'architecture technique sur lesquels le service à produire aura
    un impact
  \end{itemize}
  \item [\textbf{A1.1.3}] Étude des exigences liées à la qualité
  attendue d'un service
  \begin{itemize}
    \item \item [\textbf{C1.1.3.1}] Recenser et caractériser les
    exigences liées à la qualité attendue du service à produire
    \item [\textbf{C1.1.3.2}] Recenser et caractériser les exigences
    de sécurité pour le service à produire
  \end{itemize}
  \item [\textbf{A1.2.1}] Élaboration et présentation d'un dossier de
  choix de solution technique
  \begin{itemize}
    \item \item [\textbf{C1.2.1.1}] Recenser et caractériser des
    solutions répondant au cahier des charges (adaptation d'une
    solution existante ou réalisation d'une nouvelle)
    \item [\textbf{C1.2.1.2}] Estimer le coût d'une solution
    \item [\textbf{C1.2.1.3}] Rédiger un dossier de choix et un
    argumentaire technique
  \end{itemize}
  \item [\textbf{A1.2.2}] Rédaction des spécifications techniques de
  la solution retenue (adaptation d'une solution existante ou
  réalisation d'une nouvelle solution)
  \begin{itemize}
    \item \item [\textbf{C1.2.2.1}] Recenser les composants
    nécessaires à la réalisation de la solution retenue
    \item [\textbf{C1.2.2.2}] Décrire l'implantation des différents
    composants de la solution et les échanges entre eux
    \item [\textbf{C1.2.2.3}] Rédiger les spécifications
    fonctionnelles et techniques de la solution retenue dans le
    formalisme exigé par l'organisation
  \end{itemize}
  \item [\textbf{A1.2.3}] Évaluation des risques liés à l'utilisation
  d'un service
  \begin{itemize}
    \item \item [\textbf{C1.2.3.1}] Recenser les risques liés à une
    mauvaise utilisation ou à une utilisation malveillante du service
    \item [\textbf{C1.2.3.2}] Recenser les risques liés à un
    dysfonctionnement du service
    \item [\textbf{C1.2.3.3}] Prévoir les conséquences techniques de
    la non prise en compte d'un risque
  \end{itemize}
  \item [\textbf{A1.2.4}] Détermination des tests nécessaires à la
  validation d'un service
  \begin{itemize}
    \item \item [\textbf{C1.2.4.1}] Recenser les tests d'acceptation
    nécessaires à la validation du service et les résultats attendus
    \item [\textbf{C1.2.4.2}] Préparer les jeux d'essai et les
    procédures pour la réalisation des tests
  \end{itemize}
  \item [\textbf{A1.2.5}] Définition des niveaux d'habilitation
  associés à un service
  \begin{itemize}
    \item \item [\textbf{C1.2.5.1}] Recenser les utilisateurs du
    service, leurs rôles et leur niveau de responsabilité
    \item [\textbf{C1.2.5.2}] Recenser les ressources liées à
    l'utilisation du service
    \item [\textbf{C1.2.5.3}] Proposer les niveaux d'habilitation
    associés au service
  \end{itemize}
  \item [\textbf{A1.3.1}] Test d'intégration et d'acceptation d'un
  service
  \begin{itemize}
    \item \item [\textbf{C1.3.1.1}] Mettre en place l'environnement de
    test du service
    \item [\textbf{C1.3.1.2}] Tester le service
    \item [\textbf{C1.3.1.3}] Rédiger le rapport de test
  \end{itemize}
  \item [\textbf{A1.3.2}] Définition des éléments nécessaires à la
  continuité d'un service
  \begin{itemize}
    \item \item [\textbf{C1.3.2.1}] Identifier les éléments à
    sauvegarder et à journaliser pour assurer la continuité du service
    et la traçabilité des transactions
    \item [\textbf{C1.3.2.2}] Spécifier les procédures d'alerte
    associées au service
    \item [\textbf{C1.3.2.3}] Décrire les solutions de fonctionnement
    en mode dégradé et les procédures de reprise du service
  \end{itemize}
  \item [\textbf{A1.3.3}] Accompagnement de la mise en place d'un
  nouveau service
  \begin{itemize}
    \item \item [\textbf{C1.3.3.1}] Mettre en place l'environnement de
    formation au nouveau service
    \item [\textbf{C1.3.3.2}] Informer et former les utilisateurs
  \end{itemize}
  \item [\textbf{A1.3.4}] Déploiement d'un service
  \begin{itemize}
    \item \item [\textbf{C1.3.4.1}] Mettre au point une procédure
    d'installation de la solution
    \item [\textbf{C1.3.4.2}] Automatiser l'installation de la
    solution
    \item [\textbf{C1.3.4.3}] Mettre en exploitation le service
  \end{itemize}
  \item [\textbf{A1.4.1}] Participation à un projet
  \begin{itemize}
    \item \item [\textbf{C1.4.1.1}] Établir son planning personnel en
    fonction des exigences et du déroulement du projet
    \item [\textbf{C1.4.1.2}] Rendre compte de son activité
  \end{itemize}
  \item [\textbf{A1.4.2}] Évaluation des indicateurs de suivi d'un
  projet et justification des écarts
  \begin{itemize}
    \item \item [\textbf{C1.4.2.1}] Suivre l'exécution du projet
    \item [\textbf{C1.4.2.2}] Analyser les écarts entre temps prévu et
    temps consommé
    \item [\textbf{C1.4.2.3}] Contribuer à l'évaluation du projet
  \end{itemize}
  \item [\textbf{A1.4.3}] Gestion des ressources
  \begin{itemize}
    \item \item [\textbf{C1.4.3.1}] Recenser les ressources humaines,
    matérielles, logicielles et budgétaires nécessaires à l'exécution
    du projet et de ses tâches personnelles
    \item [\textbf{C1.4.3.2}] Recenser les ressources humaines,
    matérielles, logicielles et budgétaires nécessaires à l'exécution
    du projet et de ses tâches personnelles
  \end{itemize}
  \item [\textbf{A2.1.1}] Accompagnement des utilisateurs dans la
  prise en main d'un service
  \begin{itemize}
    \item \item [\textbf{C2.1.1.1}] Aider les utilisateurs dans
    l'appropriation du nouveau service
    \item [\textbf{C2.1.1.2}] Identifier des besoins de formation
    complémentaires
    \item [\textbf{C2.1.1.3}] Rendre compte de la satisfaction des
    utilisateurs
  \end{itemize}
  \item [\textbf{A2.1.2}] Évaluation et maintien de la qualité d'un
  service
  \begin{itemize}
    \item \item [\textbf{C2.1.2.1}] Analyser les indicateurs de
    qualité du service
    \item [\textbf{C2.1.2.2}] Appliquer les procédures d'alerte
    destinées à rétablir la qualité du service
    \item [\textbf{C2.1.2.3}] Vérifier périodiquement le
    fonctionnement du service en mode dégradé et la disponibilité des
    éléments permettant une reprise du service
    \item [\textbf{C2.1.2.4}] Superviser les services et leur
    utilisation
    \item [\textbf{C2.1.2.5}] Contrôler la confidentialité et
    l'intégrité des données
    \item [\textbf{C2.1.2.6}] Exploiter les indicateurs et les
    fichiers d'audit
    \item [\textbf{C2.1.2.7}] Produire les rapports d'activité
    demandés par les différents acteurs
  \end{itemize}
  \item [\textbf{A2.2.1}] Suivi et résolution d'incidents
  \begin{itemize}
    \item \item [\textbf{C2.2.1.1}] Résoudre l'incident en s'appuyant
    sur une base de connaissances et la documentation associée ou
    solliciter l'entité compétente
    \item [\textbf{C2.2.1.2}] Prendre le contrôle d'un système à
    distance
    \item [\textbf{C2.2.1.3}] Rédiger un rapport d'incident et
    mémoriser l'incident et sa résolution dans une base de
    connaissances
    \item [\textbf{C2.2.1.4}] Faire évoluer une procédure de
    résolution d'incident
  \end{itemize}
  \item [\textbf{A2.2.2}] Suivi et réponse à des demandes d'assistance
  \begin{itemize}
    \item \item [\textbf{C2.2.2.1}] Identifier le niveau d'assistance
    souhaité et proposer une réponse adaptée en s'appuyant sur une
    base de connaissances et sur la documentation associée ou
    solliciter l'entité compétente
    \item [\textbf{C2.2.2.2}] Informer l'utilisateur de la situation
    de sa demande
    \item [\textbf{C2.2.2.3}] Prendre le contrôle d'un poste
    utilisateur à distance
    \item [\textbf{C2.2.2.4}] Mémoriser la demande d'assistance et sa
    réponse dans une base de connaissances
  \end{itemize}
  \item [\textbf{A2.2.3}] Réponse à une interruption de service
  \begin{itemize}
    \item \item [\textbf{C2.2.3.1}] Appliquer la procédure de
    continuité du service en mode dégradé
    \item [\textbf{C2.2.3.2}] Appliquer la procédure de reprise du
    service
  \end{itemize}
  \item [\textbf{A2.3.1}] Identification, qualification et évaluation
  d'un problème
  \begin{itemize}
    \item \item [\textbf{C2.3.1.1}] Repérer une suite de
    dysfonctionnements récurrents d'un service
    \item [\textbf{C2.3.1.2}] Identifier les causes de ce
    dysfonctionnement
    \item [\textbf{C2.3.1.3}] Qualifier le problème (contexte et
    environnement)
    \item [\textbf{C2.3.1.4}] Définir le degré d'urgence du problème
    \item [\textbf{C2.3.1.5}] Évaluer les conséquences techniques du
    problème
  \end{itemize}
  \item [\textbf{A2.3.2}] Proposition d'amélioration d'un service
  \begin{itemize}
    \item \item [\textbf{C2.3.2.1}] Décrire les incidences d'un
    changement proposé sur le service
    \item [\textbf{C2.3.2.2}] Évaluer le délai et le coût de
    réalisation du changement proposé
    \item [\textbf{C2.3.2.3}] Recenser les risques techniques,
    humains, financiers et juridiques associés au changement proposé
  \end{itemize}
  \item [\textbf{A3.1.1}] Proposition d'une solution d'infrastructure
  \begin{itemize}
    \item \item [\textbf{C3.1.1.1}] Lister les composants matériels et
    logiciels nécessaires à la prise en charge des processus, des flux
    d'information et de leur rôle
    \item [\textbf{C3.1.1.2}] Caractériser les éléments
    d'interconnexion, les services, les serveurs et les équipements
    terminaux nécessaires
    \item [\textbf{C3.1.1.3}] Caractériser les éléments permettant
    d'assurer la qualité et la sécurité des services
    \item [\textbf{C3.1.1.4}] Recenser les modifications et/ou les
    acquisitions nécessaires à la mise en place d'une solution
    d'infrastructure compatible avec le budget et le planning
    prévisionnels
    \item [\textbf{C3.1.1.5}] Caractériser les solutions
    d'interconnexion utilisées entre un réseau et d'autres réseaux
    internes ou externes à l'organisation
  \end{itemize}
  \item [\textbf{A3.1.2}] Maquettage et prototypage d'une solution
  d'infrastructure
  \begin{itemize}
    \item \item [\textbf{C3.1.2.1}] Concevoir une maquette de la
    solution
    \item [\textbf{C3.1.2.2}] Construire un prototype de la solution
    \item [\textbf{C3.1.2.3}] Préparer l'intégration d'un composant
    d'infrastructure
  \end{itemize}
  \item [\textbf{A3.1.3}] Prise en compte du niveau de sécurité
  nécessaire à une infrastructure
  \begin{itemize}
    \item \item [\textbf{C3.1.3.1}] Caractériser des solutions de
    sécurité et en évaluer le coût
    \item [\textbf{C3.1.3.2}] Proposer une solution de sécurité
    compatible avec les contraintes techniques, financières,
    juridiques et organisationnelles
    \item [\textbf{C3.1.3.3}] Décrire une solution de sécurité et les
    risques couverts
  \end{itemize}
  \item [\textbf{A3.2.1}] Installation et configuration d'éléments
  d'infrastructure
  \begin{itemize}
    \item \item [\textbf{C3.2.1.1}] Installer et configurer un élément
    d'interconnexion, un service, un serveur, un équipement terminal
    utilisateur
    \item [\textbf{C3.2.1.2}] Installer et configurer un élément
    d'interconnexion, un service, un serveur, un équipement terminal
    utilisateur
    \item [\textbf{C3.2.1.3}] Installer et configurer des éléments de
    sécurité permettant d'assurer la protection du système
    informatique
  \end{itemize}
  \item [\textbf{A3.2.2}] Remplacement ou mise à jour d'éléments
  défectueux ou obsolètes
  \begin{itemize}
    \item \item [\textbf{C3.2.2.1}] Élaborer une procédure de
    remplacement ou de migration respectant la continuité d'un service
    \item [\textbf{C3.2.2.2}] Mettre en œuvre une procédure de
    remplacement ou de migration
  \end{itemize}
  \item [\textbf{A3.2.3}] Mise à jour de la documentation technique
  d'une solution d'infrastructure
  \begin{itemize}
    \item \item [\textbf{C3.2.3.1}] Repérer les éléments de la
    documentation à mettre à jour
    \item [\textbf{C3.2.3.2}] Mettre à jour la documentation
  \end{itemize}
  \item [\textbf{A3.3.1}] Administration sur site ou à distance des
  éléments d'un réseau, de serveurs, de services et d'équipements
  terminaux
  \begin{itemize}
    \item \item [\textbf{C3.3.1.1}] Installer et configurer des
    éléments d'administration sur site ou à distance
    \item [\textbf{C3.3.1.2}] Administrer des éléments
    d'infrastructure sur site ou à distance
  \end{itemize}
  \item [\textbf{A3.3.2}] Planification des sauvegardes et gestion des
  restaurations
  \begin{itemize}
    \item \item [\textbf{C3.3.2.1}] Installer et configurer des outils
    de sauvegarde et de restauration
    \item [\textbf{C3.3.2.2}] Définir des procédures de sauvegarde et
    de restauration
    \item [\textbf{C3.3.2.3}] Appliquer des procédures de sauvegarde
    et de restauration
  \end{itemize}
  \item [\textbf{A3.3.3}] Gestion des identités et des habilitations
  \begin{itemize}
    \item \item [\textbf{C3.3.3.1}] Identifier les besoins en gestion
    d'identité permettant de protéger les éléments d'une
    infrastructure
    \item [\textbf{C3.3.3.2}] Gérer des utilisateurs et une structure
    organisationnelle
    \item [\textbf{C3.3.3.3}] Affecter des droits aux utilisateurs sur
    les éléments d'une solution d'infrastructure
  \end{itemize}
  \item [\textbf{A3.3.4}] Automatisation des tâches d'administration
  \begin{itemize}
    \item \item [\textbf{C3.3.4.1}] Repérer les tâches
    d'administration à automatiser
    \item [\textbf{C3.3.4.2}] Concevoir, réaliser et mettre en place
    une procédure d'automatisation
  \end{itemize}
  \item [\textbf{A3.3.5}] Gestion des indicateurs et des fichiers
  d'activité
  \begin{itemize}
    \item \item [\textbf{C3.3.5.1}] Installer et configurer les outils
    nécessaires à la production d'indicateurs d'activité et à
    l'exploitation de fichiers d'activité
    \item [\textbf{C3.3.5.2}] Assurer la confidentialité des
    informations collectées et traitées
  \end{itemize}
  \item [\textbf{A4.1.1}] Proposition d'une solution applicative
  \begin{itemize}
    \item \item [\textbf{C4.1.1.1}] Identifier les composants
    logiciels nécessaires à la conception de la solution
    \item [\textbf{C4.1.1.2}] Estimer les éléments de coût et le délai
    de mise en œuvre de la solution
  \end{itemize}
  \item [\textbf{A4.1.2}] Conception ou adaptation de l'interface
  utilisateur d'une solution applicative
  \begin{itemize}
    \item \item [\textbf{C4.1.2.1}] Définir les spécifications de
    l'interface utilisateur de la solution applicative
    \item [\textbf{C4.1.2.2}] Maquetter un élément de la solution
    applicative
    \item [\textbf{C4.1.2.3}] Concevoir et valider la maquette en
    collaboration avec des utilisateurs
  \end{itemize}
  \item [\textbf{A4.1.3}] Conception ou adaptation d'une base de
  données
  \begin{itemize}
    \item \item [\textbf{C4.1.3.1}] Modéliser le schéma de données
    nécessaire à la mise en place de la solution applicative
    \item [\textbf{C4.1.3.2}] Implémenter le schéma de données dans un
    SGBD
    \item [\textbf{C4.1.3.3}] Programmer des éléments de la solution
    applicative dans le langage d'un SGBD
    \item [\textbf{C4.1.3.4}] Manipuler les données liées à la
    solution applicative à travers un langage de requête
  \end{itemize}
  \item [\textbf{A4.1.4}] Définition des caractéristiques d'une
  solution applicative
  \begin{itemize}
    \item \item [\textbf{C4.1.4.1}] Recenser et caractériser les
    composants existants ou à développer utiles à la réalisation de la
    solution applicative dans le respect des budgets et planning
    prévisionnels
  \end{itemize}
  \item [\textbf{A4.1.5}] Prototypage de composants logiciels
  \begin{itemize}
    \item \item [\textbf{C4.1.5.1}] Choisir les éléments de la
    solution à prototyper
    \item [\textbf{C4.1.5.2}] Développer un prototype
    \item [\textbf{C4.1.5.3}] Valider un prototype
  \end{itemize}
  \item [\textbf{A4.1.6}] Gestion d'environnements de développement et
  de test
  \begin{itemize}
    \item \item [\textbf{C4.1.6.1}] Mettre en place et exploiter un
    environnement de développement
    \item [\textbf{C4.1.6.2}] Mettre en place et exploiter un
    environnement de test
  \end{itemize}
  \item [\textbf{A4.1.7}] Développement, utilisation ou adaptation de
  composants logiciels
  \begin{itemize}
    \item \item [\textbf{C4.1.7.1}] Développer les éléments d'une
    solution
    \item [\textbf{C4.1.7.2}] Créer un composant logiciel
    \item [\textbf{C4.1.7.3}] Analyser et modifier le code d'un
    composant logiciel
    \item [\textbf{C4.1.7.4}] Utiliser des composants d'accès aux
    données
    \item [\textbf{C4.1.7.5}] Mettre en place des éléments de sécurité
    liés à l'utilisation d'un composant logiciel
  \end{itemize}
  \item [\textbf{A4.1.8}] Réalisation des tests nécessaires à la
  validation d'éléments adaptés ou développés
  \begin{itemize}
    \item \item [\textbf{C4.1.8.1}] Élaborer et réaliser des tests
    unitaires
    \item [\textbf{C4.1.8.2}] Mettre en évidence et corriger les
    écarts
  \end{itemize}
  \item [\textbf{A4.1.9}] Rédaction d'une documentation technique
  \begin{itemize}
    \item \item [\textbf{C4.1.9.1}] Produire ou mettre à jour la
    documentation technique d'une solution applicative et de ses
    composants logiciels
  \end{itemize}
  \item [\textbf{A4.1.10}] Rédaction d'une documentation d'utilisation
  \begin{itemize}
    \item \item [\textbf{C4.1.10.1}] Rédiger la documentation
    d'utilisation, une aide en ligne, une FAQ
    \item [\textbf{C4.1.10.2}] Adapter la documentation d'utilisation
    à chaque contexte d'utilisation
  \end{itemize}
  \item [\textbf{A4.2.1}] Analyse et correction d'un
  dysfonctionnement, d'un problème de qualité de service ou de
  sécurité
  \begin{itemize}
    \item \item [\textbf{C4.2.1.1}] Élaborer un jeu d'essai permettant
    de reproduire le dysfonctionnement
    \item [\textbf{C4.2.1.2}] Repérer les composants à l'origine du
    dysfonctionnement
    \item [\textbf{C4.2.1.3}] Concevoir les mises à jour à effectuer
    \item [\textbf{C4.2.1.4}] Réaliser les mises à jour
  \end{itemize}
  \item [\textbf{A4.2.2}] Adaptation d'une solution applicative aux
  évolutions de ses composants
  \begin{itemize}
    \item \item [\textbf{C4.2.2.1}] Repérer les évolutions des
    composants utilisés et leurs conséquences
    \item [\textbf{C4.2.2.2}] Concevoir les mises à jour à effectuer
    \item [\textbf{C4.2.2.3}] Élaborer et réaliser les tests unitaires
    des composants mis à jour
  \end{itemize}
  \item [\textbf{A4.2.3}] Réalisation des tests nécessaires à la mise
  en production d'éléments mis à jour
  \begin{itemize}
    \item \item [\textbf{C4.2.3.1}] Élaborer et réaliser des tests
    d'intégration et de non régression de la solution mise à jour
    \item [\textbf{C4.2.3.2}] Concevoir une procédure de migration et
    l'appliquer dans le respect de la continuité de service
  \end{itemize}
  \item [\textbf{A4.2.4}] Mise à jour d'une documentation technique
  \begin{itemize}
    \item \item [\textbf{C4.2.4.1}] Repérer les éléments de la
    documentation à mettre à jour
    \item [\textbf{C4.2.4.2}] Mettre à jour une documentation
  \end{itemize}
  \item [\textbf{A5.1.1}] Mise en place d'une gestion de configuration
  \begin{itemize}
    \item \item [\textbf{C5.1.1.1}] Recenser les caractéristiques
    techniques nécessaires à la gestion des éléments de la
    configuration d'une organisation
    \item [\textbf{C5.1.1.2}] Paramétrer une solution de gestion des
    éléments d'une configuration
  \end{itemize}
  \item [\textbf{A5.1.2}] Recueil d'informations sur une configuration
  et ses éléments
  \begin{itemize}
    \item \item [\textbf{C5.1.2.1}] Renseigner les événements relatifs
    au cycle de vie d'un élément de la configuration
    \item [\textbf{C5.1.2.2}] Actualiser les caractéristiques des
    éléments de la configuration
  \end{itemize}
  \item [\textbf{A5.1.3}] Suivi d'une configuration et de ses éléments
  \begin{itemize}
    \item \item [\textbf{C5.1.3.1}] Contrôler et auditer les éléments
    de la configuration
    \item [\textbf{C5.1.3.2}] Reconstituer un historique des
    modifications effectuées sur les éléments de la configuration
    \item [\textbf{C5.1.3.3}] Identifier les éléments de la
    configuration à modifier ou à remplacer
    \item [\textbf{C5.1.3.4}] Repérer les équipements obsolètes et en
    proposer le traitement dans le respect de la réglementation en
    vigueur
  \end{itemize}
  \item [\textbf{A5.1.4}] Étude de propositions de contrat de service
  (client, fournisseur)
  \begin{itemize}
    \item \item [\textbf{C5.1.4.1}] Assister la maîtrise d'ouvrage
    dans l'analyse technique de la proposition de contrat
    \item [\textbf{C5.1.4.2}] Interpréter des indicateurs de suivi de
    la prestation associée à la proposition de contrat
    \item [\textbf{C5.1.4.3}] Renseigner les éléments permettant
    d'estimer la valeur du service
  \end{itemize}
  \item [\textbf{A5.1.5}] Évaluation d'un élément de configuration ou
  d'une configuration
  \begin{itemize}
    \item \item [\textbf{C5.1.5.1}] Vérifier un plan d'amortissement
    \item [\textbf{C5.1.5.2}] Apprécier la valeur actuelle d'un
    élément de configuration
  \end{itemize}
  \item [\textbf{A5.1.6}] Évaluation d'un investissement informatique
  \begin{itemize}
    \item \item [\textbf{C5.1.6.1}] Renseigner les variables d'une
    étude de rentabilité d'un investissement
    \item [\textbf{C5.1.6.2}] Caractériser et prévoir les
    investissements matériels et logiciels
  \end{itemize}
  \item [\textbf{A5.2.1}] Exploitation des référentiels, normes et
  standards adoptés par le prestataire informatique
  \begin{itemize}
    \item \item [\textbf{C5.2.1.1}] Évaluer le degré de conformité des
    pratiques à un référentiel, à une norme ou à un standard adopté
    par le prestataire informatique
    \item [\textbf{C5.2.1.2}] Identifier et partager les bonnes
    pratiques à intégrer
  \end{itemize}
  \item [\textbf{A5.2.2}] Veille technologique
  \begin{itemize}
    \item \item [\textbf{C5.2.2.1}] Définir une stratégie de recherche
    d'informations
    \item [\textbf{C5.2.2.2}] Tenir à jour une liste de sources
    d'information
    \item [\textbf{C5.2.2.3}] Évaluer la qualité d'une source
    d'information en fonction d'un besoin
    \item [\textbf{C5.2.2.4}] Synthétiser et diffuser les résultats
    d'une veille
  \end{itemize}
  \item [\textbf{A5.2.3}] Repérage des compléments de formation ou
  d'auto-formation utiles à l'acquisition de nouvelles compétences
  \begin{itemize}
    \item \item [\textbf{C5.2.3.1}] Identifier les besoins de
    formation pour mettre en œuvre une technologie, un composant, un
    outil ou une méthode
    \item [\textbf{C5.2.3.2}] Repérer l'offre et les dispositifs de
    formation
  \end{itemize}
  \item [\textbf{A5.2.4}] Étude d'une technologie, d'un composant,
  d'un outil ou d'une méthode
  \begin{itemize}
    \item \item [\textbf{C5.2.4.1}] Se documenter à propos d'une
    technologie, d'un composant, d'un outil ou d'une méthode
    \item [\textbf{C5.2.4.2}] Identifier le potentiel et les limites
    d'une technologie, d'un composant, d'un outil ou d'une méthode par
    rapport à un service à produire
  \end{itemize}
\end{enumerate}

\clearpage%
% References bibliographiques
\printbibheading%
\printbibliography[nottype=online,check=notonline,heading=subbibliography,title={Bibliographiques}]
\printbibliography[check=online,heading=subbibliography,title={Webographiques}]

\printglossaries%

\end{document}
